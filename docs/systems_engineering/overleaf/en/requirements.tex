\documentclass[11pt,a4paper]{article}
\usepackage[utf8]{inputenc}
\usepackage[T1]{fontenc}
\usepackage[english]{babel}
\usepackage{geometry}
\usepackage{longtable}
\usepackage{booktabs}
\usepackage{hyperref}
\geometry{margin=2.5cm}

\title{System Requirements and Systems Engineering Documentation\\AymaraLang (SEBOK/INCOSE)}
\author{AymaraLang Project}
\date{\today}

\begin{document}
\maketitle

\section{Purpose}
Define the requirements for the \textbf{AymaraLang (aym)} system and its \textbf{aymc} compiler, as well as the operational context, stakeholders, interfaces, constraints, and verification criteria, following SEBOK and INCOSE practices.

\section{System scope}
\textbf{System of interest (SoS):}
\begin{itemize}
    \item \texttt{aymc} compiler (C++17) and executable generation backend.
    \item Interpreter (REPL mode and AST execution).
    \item AymaraLang language (\texttt{.aym} source files).
    \item Minimal runtime for I/O and standard utilities.
\end{itemize}

\textbf{Out of scope:}
\begin{itemize}
    \item IDEs or editor extensions.
    \item Additional external standard libraries.
    \item Integration with cloud services.
\end{itemize}

\section{Stakeholders and needs}
\begin{longtable}{@{}p{4cm}p{8cm}p{2cm}@{}}
\toprule
\textbf{Stakeholder} & \textbf{Need/goal} & \textbf{Priority} \\ \midrule
Teachers and students & Accessible language representative of Aymara & High \\
Compiler developers & Modular, maintainable architecture & High \\
End users & Fast native executables & High \\
Cultural community & Language preservation and educational materials & Medium \\ \bottomrule
\end{longtable}

\section{Concept of operations (ConOps)}
\begin{enumerate}
    \item The user writes \texttt{.aym} code with Aymara syntax.
    \item They run \texttt{aymc} to compile into a native binary (\texttt{.ayn} or \texttt{.exe}).
    \item The compiler performs: lexer $\rightarrow$ parser $\rightarrow$ AST $\rightarrow$ semantic analysis $\rightarrow$ codegen $\rightarrow$ assembly/linking.
    \item The resulting executable runs on the target system.
    \item Optionally, the user uses REPL mode for interactive execution.
\end{enumerate}

\section{System context and boundaries}
\textbf{Operational environment:}
\begin{itemize}
    \item Linux/Windows systems (with toolchain support for NASM/LD/GCC or MinGW).
    \item Build dependencies: \texttt{g++}/\texttt{clang++}, \texttt{nasm}, \texttt{gcc/ld}.
\end{itemize}

\textbf{Key external interfaces:}
\begin{itemize}
    \item Compiler CLI (\texttt{aymc}).
    \item File system (read \texttt{.aym}, write \texttt{.asm}/\texttt{.ll} and binaries).
    \item LLVM backend (optional, enabled with \texttt{--llvm}).
\end{itemize}

\section{Functional architecture (summary)}
\begin{longtable}{@{}p{4cm}p{5cm}p{5cm}@{}}
\toprule
\textbf{Function} & \textbf{Input} & \textbf{Output} \\ \midrule
Lexical analysis & \texttt{.aym} code & Tokens \\
Syntax analysis & Tokens & AST \\
Semantic analysis & AST & Validated AST + symbols \\
Code generation & Validated AST & \texttt{.asm}/\texttt{.ll} \\
Assembly/linking & \texttt{.asm} & Native binary \\
REPL/Interpreter & Line of code & Console result \\ \bottomrule
\end{longtable}

\section{System requirements}

\subsection{Functional requirements (FR)}
\begin{itemize}
    \item \textbf{FR-01} The system shall accept one or more \texttt{.aym} files as CLI input.
    \item \textbf{FR-02} The system shall tokenize source code and produce tokens according to language keywords and symbols.
    \item \textbf{FR-03} The system shall generate a valid AST via LL syntactic analysis.
    \item \textbf{FR-04} The system shall validate semantics (types, symbols, and calls) before code generation.
    \item \textbf{FR-05} The system shall generate NASM x86\_64 assembly and link a native executable.
    \item \textbf{FR-06} The system shall allow selecting a Linux or Windows target via CLI.
    \item \textbf{FR-07} The system shall provide REPL mode for interactive execution.
    \item \textbf{FR-08} The system shall support an experimental LLVM backend that outputs \texttt{.ll}.
    \item \textbf{FR-09} The system shall support imported modules from external files.
    \item \textbf{FR-10} The system shall include built-in functions (I/O, math, arrays, randomness).
\end{itemize}

\subsection{Non-functional requirements (NFR)}
\begin{itemize}
    \item \textbf{NFR-01} The compiler shall run on Linux and Windows with a standard toolchain.
    \item \textbf{NFR-02} Compilation time shall be suitable for educational programs (\textless= 2s for small files on classroom machines).
    \item \textbf{NFR-03} The system shall provide readable error messages for lexical, syntactic, and semantic errors.
    \item \textbf{NFR-04} The codebase shall maintain a modular structure (lexer/parser/ast/semantic/codegen).
    \item \textbf{NFR-05} The project shall compile with C++17.
\end{itemize}

\subsection{Interface requirements (IR)}
\begin{itemize}
    \item \textbf{IR-01} Primary CLI: \texttt{aymc [options] file.aym ...}.
    \item \textbf{IR-02} Options: \texttt{--repl}, \texttt{--debug}, \texttt{--dump-ast}, \texttt{--seed}, \texttt{--llvm}, \texttt{--windows}, \texttt{--linux}, \texttt{-o}.
    \item \textbf{IR-03} Output files: \texttt{.asm} (NASM), \texttt{.ll} (LLVM), and native executable.
\end{itemize}

\section{Constraints and assumptions}
\begin{itemize}
    \item The toolchain (NASM and GCC/LD or MinGW) must be available in PATH.
    \item The LLVM backend is optional and depends on a build with support enabled.
    \item The language is designed for static typing and constructs defined in the grammar.
\end{itemize}

\end{document}
