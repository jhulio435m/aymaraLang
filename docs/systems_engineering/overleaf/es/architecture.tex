\documentclass[11pt,a4paper]{article}
\usepackage[utf8]{inputenc}
\usepackage[T1]{fontenc}
\usepackage[spanish]{babel}
\usepackage{geometry}
\usepackage{longtable}
\usepackage{booktabs}
\geometry{margin=2.5cm}

\title{Descripción de Arquitectura del Sistema\\AymaraLang}
\author{Proyecto AymaraLang}
\date{\today}

\begin{document}
\maketitle

\section{Visión general}
AymaraLang está compuesto por un compilador modular (\texttt{aymc}), un runtime mínimo y herramientas de apoyo. La arquitectura sigue el flujo clásico de compiladores con módulos desacoplados.

\section{Arquitectura funcional}
\begin{longtable}{@{}p{3cm}p{4cm}p{4cm}p{4cm}@{}}
\toprule
\textbf{Módulo} & \textbf{Responsabilidad} & \textbf{Entradas} & \textbf{Salidas} \\ \midrule
Lexer & Tokenización de código fuente & \texttt{.aym} & Tokens \\
Parser & Construcción de AST & Tokens & AST \\
Semantic & Análisis semántico y tipos & AST & AST validado + símbolos \\
Codegen & Generación de NASM/LLVM & AST validado & \texttt{.asm}/\texttt{.ll} \\
Linker & Ensamblado/enlace & \texttt{.asm} & Binario \\
\bottomrule
\end{longtable}

\section{Arquitectura física (deployment)}
\begin{itemize}
    \item \textbf{Entorno local:} Linux/Windows.
    \item \textbf{Dependencias:} NASM, GCC/LD o MinGW, compilador C++17.
    \item \textbf{Artefactos:} \texttt{.asm}, \texttt{.ll}, binario nativo, logs de error.
\end{itemize}

\section{Interfaces clave}
\begin{itemize}
    \item \textbf{CLI:} \texttt{aymc [opciones] archivo.aym}.
    \item \textbf{Sistema de archivos:} lectura de fuentes, escritura de artefactos.
    \item \textbf{Runtime:} funciones básicas de E/S utilizadas por el ejecutable.
\end{itemize}

\section{Datos e información}
\begin{itemize}
    \item \textbf{Tokens:} tipo, texto, línea y columna.
    \item \textbf{AST:} nodos de expresiones y sentencias.
    \item \textbf{Símbolos:} tablas de alcance para variables y funciones.
\end{itemize}

\section{Restricciones de diseño}
\begin{itemize}
    \item C++17 como estándar base.
    \item Codegen x86\_64 (NASM) como backend principal.
    \item LLVM como backend experimental.
\end{itemize}

\end{document}
