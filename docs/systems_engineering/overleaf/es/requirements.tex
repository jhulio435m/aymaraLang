\documentclass[11pt,a4paper]{article}
\usepackage[utf8]{inputenc}
\usepackage[T1]{fontenc}
\usepackage[spanish]{babel}
\usepackage{geometry}
\usepackage{longtable}
\usepackage{booktabs}
\usepackage{hyperref}
\geometry{margin=2.5cm}

\title{Especificación de requisitos y documentación de ingeniería de sistemas\\AymaraLang (SEBOK/INCOSE)}
\author{Proyecto AymaraLang}
\date{\today}

\begin{document}
\maketitle

\section{Propósito}
Definir los requisitos del sistema \textbf{AymaraLang (aym)} y su compilador \textbf{aymc}, así como el contexto operativo, stakeholders, interfaces, restricciones y criterios de verificación, siguiendo prácticas de SEBOK e INCOSE.

\section{Alcance del sistema}
\textbf{Sistema bajo estudio (SoS):}
\begin{itemize}
    \item Compilador \texttt{aymc} (C++17) y backend de generación de ejecutables.
    \item Intérprete (modo REPL y ejecución por AST).
    \item Lenguaje AymaraLang (archivos fuente \texttt{.aym}).
    \item Runtime mínimo para E/S y utilidades estándar.
\end{itemize}

\textbf{Fuera de alcance:}
\begin{itemize}
    \item IDEs o extensiones de editor.
    \item Librerías estándar externas adicionales.
    \item Integración con servicios en la nube.
\end{itemize}

\section{Stakeholders y necesidades}
\begin{longtable}{@{}p{4cm}p{8cm}p{2cm}@{}}
\toprule
\textbf{Stakeholder} & \textbf{Necesidad/objetivo} & \textbf{Prioridad} \\ \midrule
Docentes y estudiantes & Lenguaje accesible y representativo del aymara & Alta \\
Desarrolladores del compilador & Arquitectura modular y mantenible & Alta \\
Usuarios finales & Ejecutables nativos rápidos & Alta \\
Comunidad cultural & Preservación lingüística y material educativo & Media \\ \bottomrule
\end{longtable}

\section{Concepto de operaciones (ConOps)}
\begin{enumerate}
    \item El usuario escribe código \texttt{.aym} con sintaxis en aymara.
    \item Ejecuta \texttt{aymc} para compilar a un binario nativo (\texttt{.ayn} o \texttt{.exe}).
    \item El compilador realiza: lexer $\rightarrow$ parser $\rightarrow$ AST $\rightarrow$ análisis semántico $\rightarrow$ codegen $\rightarrow$ ensamblado/enlace.
    \item El ejecutable resultante corre en el sistema objetivo.
    \item Opcionalmente, el usuario utiliza el modo REPL para ejecución interactiva.
\end{enumerate}

\section{Contexto del sistema y límites}
\textbf{Entorno operativo:}
\begin{itemize}
    \item Sistemas Linux/Windows (con soporte de toolchain para NASM/LD/GCC o MinGW).
    \item Dependencias de compilación: \texttt{g++}/\texttt{clang++}, \texttt{nasm}, \texttt{gcc/ld}.
\end{itemize}

\textbf{Interfaces externas clave:}
\begin{itemize}
    \item CLI del compilador (\texttt{aymc}).
    \item Sistema de archivos (lectura \texttt{.aym}, escritura \texttt{.asm}/\texttt{.ll} y binarios).
    \item Backend LLVM (opcional, activado con \texttt{--llvm}).
\end{itemize}

\section{Arquitectura funcional (resumen)}
\begin{longtable}{@{}p{4cm}p{5cm}p{5cm}@{}}
\toprule
\textbf{Función} & \textbf{Entrada} & \textbf{Salida} \\ \midrule
Análisis léxico & Código \texttt{.aym} & Tokens \\
Análisis sintáctico & Tokens & AST \\
Análisis semántico & AST & AST validado + símbolos \\
Generación de código & AST validado & \texttt{.asm}/\texttt{.ll} \\
Ensamblado/enlace & \texttt{.asm} & Binario nativo \\
REPL/Intérprete & Línea de código & Resultado en consola \\ \bottomrule
\end{longtable}

\section{Requisitos del sistema}

\subsection{Requisitos funcionales (FR)}
\begin{itemize}
    \item \textbf{FR-01} El sistema debe aceptar uno o más archivos \texttt{.aym} como entrada por CLI.
    \item \textbf{FR-02} El sistema debe tokenizar el código fuente y producir tokens conforme a palabras clave y símbolos del lenguaje.
    \item \textbf{FR-03} El sistema debe generar un AST válido mediante análisis sintáctico LL.
    \item \textbf{FR-04} El sistema debe validar semántica (tipado, símbolos y llamadas) antes de generar código.
    \item \textbf{FR-05} El sistema debe generar ensamblador NASM x86\_64 y enlazar un ejecutable nativo.
    \item \textbf{FR-06} El sistema debe permitir seleccionar objetivo Linux o Windows desde la CLI.
    \item \textbf{FR-07} El sistema debe ofrecer modo REPL para ejecución interactiva.
    \item \textbf{FR-08} El sistema debe admitir un backend LLVM experimental que genere \texttt{.ll}.
    \item \textbf{FR-09} El sistema debe soportar módulos importados desde archivos externos.
    \item \textbf{FR-10} El sistema debe incluir funciones integradas (E/S, matemáticas, arreglos, aleatoriedad).
\end{itemize}

\subsection{Requisitos no funcionales (NFR)}
\begin{itemize}
    \item \textbf{NFR-01} El compilador debe ejecutarse en Linux y Windows con toolchain estándar.
    \item \textbf{NFR-02} El tiempo de compilación debe ser adecuado para programas educativos (\textless= 2s para archivos pequeños en equipos de clase).
    \item \textbf{NFR-03} El sistema debe proporcionar mensajes de error legibles para errores léxicos, sintácticos y semánticos.
    \item \textbf{NFR-04} El código debe mantener una estructura modular (lexer/parser/ast/semantic/codegen).
    \item \textbf{NFR-05} El proyecto debe compilarse con C++17.
\end{itemize}

\subsection{Requisitos de interfaz (IR)}
\begin{itemize}
    \item \textbf{IR-01} CLI principal: \texttt{aymc [opciones] archivo.aym ...}.
    \item \textbf{IR-02} Opciones: \texttt{--repl}, \texttt{--debug}, \texttt{--dump-ast}, \texttt{--seed}, \texttt{--llvm}, \texttt{--windows}, \texttt{--linux}, \texttt{-o}.
    \item \textbf{IR-03} Archivos de salida: \texttt{.asm} (NASM), \texttt{.ll} (LLVM), y ejecutable nativo.
\end{itemize}

\section{Restricciones y supuestos}
\begin{itemize}
    \item Se requiere toolchain (NASM y GCC/LD o MinGW) disponible en PATH.
    \item El backend LLVM es opcional y depende de compilación con soporte.
    \item El lenguaje está orientado a tipado estático y a constructs definidos en la gramática.
\end{itemize}

\section{Verificación y validación (V\&V)}
\begin{longtable}{@{}p{3cm}p{3cm}p{8cm}@{}}
\toprule
\textbf{Requisito} & \textbf{Método} & \textbf{Evidencia esperada} \\ \midrule
FR-01 & Inspección/Prueba & \texttt{aymc archivo.aym} compila sin errores \\
FR-05 & Prueba & Se genera \texttt{.asm} y binario ejecutable \\
FR-07 & Demostración & REPL ejecuta expresiones interactivas \\
FR-08 & Prueba & Se genera archivo \texttt{.ll} \\
NFR-05 & Inspección & Uso de \texttt{-std=c++17} en build \\ \bottomrule
\end{longtable}

\section{Trazabilidad (resumen)}
\begin{longtable}{@{}p{3cm}p{5cm}p{6cm}@{}}
\toprule
\textbf{Requisito} & \textbf{Elemento de diseño} & \textbf{Evidencia técnica} \\ \midrule
FR-01 & CLI \texttt{main.cpp} & Parsing de argumentos \\
FR-02 & \texttt{lexer/} & Tokenización \\
FR-03 & \texttt{parser/} & Construcción de AST \\
FR-04 & \texttt{semantic/} & Análisis de tipos \\
FR-05 & \texttt{codegen/} & Generación NASM \\
FR-07 & \texttt{interpreter/} & REPL \\
FR-08 & \texttt{codegen/llvm/} & LLVM IR \\
FR-09 & \texttt{module\_resolver} & Resolución de módulos \\
FR-10 & \texttt{builtins/} & Funciones integradas \\ \bottomrule
\end{longtable}

\section{Riesgos técnicos (alto nivel)}
\begin{itemize}
    \item Dependencia de toolchain externo para ensamblado/enlace.
    \item Backend LLVM incompleto o experimental.
    \item Compatibilidad limitada con arquitecturas distintas de x86\_64.
\end{itemize}

\section{Glosario}
\begin{itemize}
    \item \textbf{AST:} Árbol de Sintaxis Abstracta.
    \item \textbf{REPL:} Read--Eval--Print Loop, modo interactivo.
    \item \textbf{V\&V:} Verificación y validación.
    \item \textbf{ConOps:} Concepto de operaciones.
\end{itemize}

\end{document}
