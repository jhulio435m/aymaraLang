\documentclass[11pt,a4paper]{article}
\usepackage[utf8]{inputenc}
\usepackage[T1]{fontenc}
\usepackage[spanish]{babel}
\usepackage{geometry}
\usepackage{longtable}
\usepackage{booktabs}
\geometry{margin=2.5cm}

\title{Concepto de Operaciones (ConOps)\\AymaraLang}
\author{Proyecto AymaraLang}
\date{\today}

\begin{document}
\maketitle

\section{Propósito}
Describir cómo los usuarios interactúan con AymaraLang y su compilador \texttt{aymc} en escenarios reales, incluyendo flujos operativos, roles y restricciones.

\section{Escenarios operativos}
\subsection{Uso educativo en aula}
\begin{enumerate}
    \item Docente distribuye ejemplos \texttt{.aym}.
    \item Estudiantes editan el código en un editor de texto.
    \item Ejecutan \texttt{aymc} para compilar y correr los binarios.
    \item Se analiza salida para reforzar conceptos de programación.
\end{enumerate}

\subsection{Uso comunitario y cultural}
\begin{enumerate}
    \item Creación de materiales educativos en aymara.
    \item Publicación de ejemplos y ejercicios.
    \item Ejecución en equipos modestos sin depender de conexión a internet.
\end{enumerate}

\section{Actores y responsabilidades}
\begin{longtable}{@{}p{4cm}p{9cm}@{}}
\toprule
\textbf{Actor} & \textbf{Responsabilidad} \\ \midrule
Usuario final & Escribir y ejecutar programas \texttt{.aym} \\
Docente & Diseñar ejercicios y guiar el aprendizaje \\
Desarrollador del compilador & Mantener el compilador y documentación \\ \bottomrule
\end{longtable}

\section{Supuestos}
\begin{itemize}
    \item El entorno cuenta con compilador C++ y herramientas de ensamblado/enlace.
    \item El usuario dispone de acceso de lectura/escritura al sistema de archivos.
\end{itemize}

\section{Limitaciones operativas}
\begin{itemize}
    \item El backend LLVM es opcional y puede no estar disponible.
    \item Soporte limitado a x86\_64 en la generación nativa.
\end{itemize}

\section{Criterios de éxito operacional}
\begin{itemize}
    \item Compilación exitosa en menos de 2 segundos para ejemplos educativos.
    \item Ejecución consistente en Windows y Linux.
    \item Mensajes de error comprensibles para estudiantes.
\end{itemize}

\end{document}
